%==============================================================================
% QSol-Standard-ZKProofs-v1.0
% Canonical Macros and Institutional Styling
%------------------------------------------------------------------------------
% This file defines all macros, theorem environments, and notation used
% throughout the Standard. It MUST be included via %==============================================================================
% QSol-Standard-ZKProofs-v1.0
% Canonical Macros and Institutional Styling
%------------------------------------------------------------------------------
% This file defines all macros, theorem environments, and notation used
% throughout the Standard. It MUST be included via %==============================================================================
% QSol-Standard-ZKProofs-v1.0
% Canonical Macros and Institutional Styling
%------------------------------------------------------------------------------
% This file defines all macros, theorem environments, and notation used
% throughout the Standard. It MUST be included via %==============================================================================
% QSol-Standard-ZKProofs-v1.0
% Canonical Macros and Institutional Styling
%------------------------------------------------------------------------------
% This file defines all macros, theorem environments, and notation used
% throughout the Standard. It MUST be included via \input{macros} in main.tex.
%==============================================================================

%------------------------------------------------------------------------------
% Theorem Environments (Deterministic, QSol-Style)
%------------------------------------------------------------------------------

\theoremstyle{plain}
\newtheorem{theorem}{Theorem}[section]

\theoremstyle{definition}
\newtheorem{definition}{Definition}[section]

\theoremstyle{remark}
\newtheorem{remark}{Remark}[section]

\theoremstyle{plain}
\newtheorem{lemma}{Lemma}[section]

\theoremstyle{plain}
\newtheorem{proposition}{Proposition}[section]

\theoremstyle{plain}
\newtheorem{corollary}{Corollary}[section]

%------------------------------------------------------------------------------
% QSol Institutional Math Macros
%------------------------------------------------------------------------------

% Sets and fields
\newcommand{\Z}{\mathbb{Z}}
\newcommand{\N}{\mathbb{N}}
\newcommand{\F}{\mathbb{F}}
\newcommand{\R}{\mathbb{R}}

% Groups
\newcommand{\G}{\mathbb{G}}
\newcommand{\Hh}{\mathbb{H}}

% Vectors and matrices
\newcommand{\vecx}{\mathbf{x}}
\newcommand{\vecy}{\mathbf{y}}
\newcommand{\vecz}{\mathbf{z}}
\newcommand{\matA}{\mathbf{A}}
\newcommand{\matB}{\mathbf{B}}

% Cryptographic primitives
\newcommand{\Commit}{\mathsf{Commit}}
\newcommand{\Prove}{\mathsf{Prove}}
\newcommand{\Verify}{\mathsf{Verify}}
\newcommand{\Transcript}{\mathsf{Transcript}}

% Kernels (AK, CK, SK)
\newcommand{\AK}{\mathsf{AK}}
\newcommand{\CK}{\mathsf{CK}}
\newcommand{\SK}{\mathsf{SK}}

% Profiles
\newcommand{\Profile}{\mathsf{Profile}}

%------------------------------------------------------------------------------
% Operators and Formatting Helpers
%------------------------------------------------------------------------------

\DeclareMathOperator{\poly}{poly}
\DeclareMathOperator{\negl}{negl}
\DeclareMathOperator{\rank}{rank}
\DeclareMathOperator{\Span}{span}

% Norms
\newcommand{\norm}[1]{\left\lVert #1 \right\rVert}

% Inner product
\newcommand{\inner}[2]{\left\langle #1,\, #2 \right\rangle}

% Brackets
\newcommand{\braces}[1]{\left\{ #1 \right\}}
\newcommand{\brackets}[1]{\left[ #1 \right]}
\newcommand{\parens}[1]{\left( #1 \right)}

%------------------------------------------------------------------------------
% QSol Institutional Formatting
%------------------------------------------------------------------------------

% Inline code / identifiers
\newcommand{\code}[1]{\texttt{#1}}

% Emphasized institutional terms
\newcommand{\term}[1]{\textbf{\textsc{#1}}}

% Canonical reference to the Standard
\newcommand{\QSolStandard}{\textbf{QSol-Standard-ZKProofs-v1.0}}

%------------------------------------------------------------------------------
% Environment Shortcuts
%------------------------------------------------------------------------------

\newenvironment{qsolbox}
{
  \begin{center}
  \begin{minipage}{0.92\linewidth}
  \hrule\vspace{0.75em}
}
{
  \vspace{0.75em}\hrule
  \end{minipage}
  \end{center}
}

%------------------------------------------------------------------------------
% End of Macros
%==============================================================================


 in main.tex.
%==============================================================================

%------------------------------------------------------------------------------
% Theorem Environments (Deterministic, QSol-Style)
%------------------------------------------------------------------------------

\theoremstyle{plain}
\newtheorem{theorem}{Theorem}[section]

\theoremstyle{definition}
\newtheorem{definition}{Definition}[section]

\theoremstyle{remark}
\newtheorem{remark}{Remark}[section]

\theoremstyle{plain}
\newtheorem{lemma}{Lemma}[section]

\theoremstyle{plain}
\newtheorem{proposition}{Proposition}[section]

\theoremstyle{plain}
\newtheorem{corollary}{Corollary}[section]

%------------------------------------------------------------------------------
% QSol Institutional Math Macros
%------------------------------------------------------------------------------

% Sets and fields
\newcommand{\Z}{\mathbb{Z}}
\newcommand{\N}{\mathbb{N}}
\newcommand{\F}{\mathbb{F}}
\newcommand{\R}{\mathbb{R}}

% Groups
\newcommand{\G}{\mathbb{G}}
\newcommand{\Hh}{\mathbb{H}}

% Vectors and matrices
\newcommand{\vecx}{\mathbf{x}}
\newcommand{\vecy}{\mathbf{y}}
\newcommand{\vecz}{\mathbf{z}}
\newcommand{\matA}{\mathbf{A}}
\newcommand{\matB}{\mathbf{B}}

% Cryptographic primitives
\newcommand{\Commit}{\mathsf{Commit}}
\newcommand{\Prove}{\mathsf{Prove}}
\newcommand{\Verify}{\mathsf{Verify}}
\newcommand{\Transcript}{\mathsf{Transcript}}

% Kernels (AK, CK, SK)
\newcommand{\AK}{\mathsf{AK}}
\newcommand{\CK}{\mathsf{CK}}
\newcommand{\SK}{\mathsf{SK}}

% Profiles
\newcommand{\Profile}{\mathsf{Profile}}

%------------------------------------------------------------------------------
% Operators and Formatting Helpers
%------------------------------------------------------------------------------

\DeclareMathOperator{\poly}{poly}
\DeclareMathOperator{\negl}{negl}
\DeclareMathOperator{\rank}{rank}
\DeclareMathOperator{\Span}{span}

% Norms
\newcommand{\norm}[1]{\left\lVert #1 \right\rVert}

% Inner product
\newcommand{\inner}[2]{\left\langle #1,\, #2 \right\rangle}

% Brackets
\newcommand{\braces}[1]{\left\{ #1 \right\}}
\newcommand{\brackets}[1]{\left[ #1 \right]}
\newcommand{\parens}[1]{\left( #1 \right)}

%------------------------------------------------------------------------------
% QSol Institutional Formatting
%------------------------------------------------------------------------------

% Inline code / identifiers
\newcommand{\code}[1]{\texttt{#1}}

% Emphasized institutional terms
\newcommand{\term}[1]{\textbf{\textsc{#1}}}

% Canonical reference to the Standard
\newcommand{\QSolStandard}{\textbf{QSol-Standard-ZKProofs-v1.0}}

%------------------------------------------------------------------------------
% Environment Shortcuts
%------------------------------------------------------------------------------

\newenvironment{qsolbox}
{
  \begin{center}
  \begin{minipage}{0.92\linewidth}
  \hrule\vspace{0.75em}
}
{
  \vspace{0.75em}\hrule
  \end{minipage}
  \end{center}
}

%------------------------------------------------------------------------------
% End of Macros
%==============================================================================


 in main.tex.
%==============================================================================

%------------------------------------------------------------------------------
% Theorem Environments (Deterministic, QSol-Style)
%------------------------------------------------------------------------------

\theoremstyle{plain}
\newtheorem{theorem}{Theorem}[section]

\theoremstyle{definition}
\newtheorem{definition}{Definition}[section]

\theoremstyle{remark}
\newtheorem{remark}{Remark}[section]

\theoremstyle{plain}
\newtheorem{lemma}{Lemma}[section]

\theoremstyle{plain}
\newtheorem{proposition}{Proposition}[section]

\theoremstyle{plain}
\newtheorem{corollary}{Corollary}[section]

%------------------------------------------------------------------------------
% QSol Institutional Math Macros
%------------------------------------------------------------------------------

% Sets and fields
\newcommand{\Z}{\mathbb{Z}}
\newcommand{\N}{\mathbb{N}}
\newcommand{\F}{\mathbb{F}}
\newcommand{\R}{\mathbb{R}}

% Groups
\newcommand{\G}{\mathbb{G}}
\newcommand{\Hh}{\mathbb{H}}

% Vectors and matrices
\newcommand{\vecx}{\mathbf{x}}
\newcommand{\vecy}{\mathbf{y}}
\newcommand{\vecz}{\mathbf{z}}
\newcommand{\matA}{\mathbf{A}}
\newcommand{\matB}{\mathbf{B}}

% Cryptographic primitives
\newcommand{\Commit}{\mathsf{Commit}}
\newcommand{\Prove}{\mathsf{Prove}}
\newcommand{\Verify}{\mathsf{Verify}}
\newcommand{\Transcript}{\mathsf{Transcript}}

% Kernels (AK, CK, SK)
\newcommand{\AK}{\mathsf{AK}}
\newcommand{\CK}{\mathsf{CK}}
\newcommand{\SK}{\mathsf{SK}}

% Profiles
\newcommand{\Profile}{\mathsf{Profile}}

%------------------------------------------------------------------------------
% Operators and Formatting Helpers
%------------------------------------------------------------------------------

\DeclareMathOperator{\poly}{poly}
\DeclareMathOperator{\negl}{negl}
\DeclareMathOperator{\rank}{rank}
\DeclareMathOperator{\Span}{span}

% Norms
\newcommand{\norm}[1]{\left\lVert #1 \right\rVert}

% Inner product
\newcommand{\inner}[2]{\left\langle #1,\, #2 \right\rangle}

% Brackets
\newcommand{\braces}[1]{\left\{ #1 \right\}}
\newcommand{\brackets}[1]{\left[ #1 \right]}
\newcommand{\parens}[1]{\left( #1 \right)}

%------------------------------------------------------------------------------
% QSol Institutional Formatting
%------------------------------------------------------------------------------

% Inline code / identifiers
\newcommand{\code}[1]{\texttt{#1}}

% Emphasized institutional terms
\newcommand{\term}[1]{\textbf{\textsc{#1}}}

% Canonical reference to the Standard
\newcommand{\QSolStandard}{\textbf{QSol-Standard-ZKProofs-v1.0}}

%------------------------------------------------------------------------------
% Environment Shortcuts
%------------------------------------------------------------------------------

\newenvironment{qsolbox}
{
  \begin{center}
  \begin{minipage}{0.92\linewidth}
  \hrule\vspace{0.75em}
}
{
  \vspace{0.75em}\hrule
  \end{minipage}
  \end{center}
}

%------------------------------------------------------------------------------
% End of Macros
%==============================================================================


 in main.tex.
%==============================================================================

%------------------------------------------------------------------------------
% Theorem Environments (Deterministic, QSol-Style)
%------------------------------------------------------------------------------

\theoremstyle{plain}
\newtheorem{theorem}{Theorem}[section]

\theoremstyle{definition}
\newtheorem{definition}{Definition}[section]

\theoremstyle{remark}
\newtheorem{remark}{Remark}[section]

\theoremstyle{plain}
\newtheorem{lemma}{Lemma}[section]

\theoremstyle{plain}
\newtheorem{proposition}{Proposition}[section]

\theoremstyle{plain}
\newtheorem{corollary}{Corollary}[section]

%------------------------------------------------------------------------------
% QSol Institutional Math Macros
%------------------------------------------------------------------------------

% Sets and fields
\newcommand{\Z}{\mathbb{Z}}
\newcommand{\N}{\mathbb{N}}
\newcommand{\F}{\mathbb{F}}
\newcommand{\R}{\mathbb{R}}

% Groups
\newcommand{\G}{\mathbb{G}}
\newcommand{\Hh}{\mathbb{H}}

% Vectors and matrices
\newcommand{\vecx}{\mathbf{x}}
\newcommand{\vecy}{\mathbf{y}}
\newcommand{\vecz}{\mathbf{z}}
\newcommand{\matA}{\mathbf{A}}
\newcommand{\matB}{\mathbf{B}}

% Cryptographic primitives
\newcommand{\Commit}{\mathsf{Commit}}
\newcommand{\Prove}{\mathsf{Prove}}
\newcommand{\Verify}{\mathsf{Verify}}
\newcommand{\Transcript}{\mathsf{Transcript}}

% Kernels (AK, CK, SK)
\newcommand{\AK}{\mathsf{AK}}
\newcommand{\CK}{\mathsf{CK}}
\newcommand{\SK}{\mathsf{SK}}

% Profiles
\newcommand{\Profile}{\mathsf{Profile}}

%------------------------------------------------------------------------------
% Operators and Formatting Helpers
%------------------------------------------------------------------------------

\DeclareMathOperator{\poly}{poly}
\DeclareMathOperator{\negl}{negl}
\DeclareMathOperator{\rank}{rank}
\DeclareMathOperator{\Span}{span}

% Norms
\newcommand{\norm}[1]{\left\lVert #1 \right\rVert}

% Inner product
\newcommand{\inner}[2]{\left\langle #1,\, #2 \right\rangle}

% Brackets
\newcommand{\braces}[1]{\left\{ #1 \right\}}
\newcommand{\brackets}[1]{\left[ #1 \right]}
\newcommand{\parens}[1]{\left( #1 \right)}

%------------------------------------------------------------------------------
% QSol Institutional Formatting
%------------------------------------------------------------------------------

% Inline code / identifiers
\newcommand{\code}[1]{\texttt{#1}}

% Emphasized institutional terms
\newcommand{\term}[1]{\textbf{\textsc{#1}}}

% Canonical reference to the Standard
\newcommand{\QSolStandard}{\textbf{QSol-Standard-ZKProofs-v1.0}}

%------------------------------------------------------------------------------
% Environment Shortcuts
%------------------------------------------------------------------------------

\newenvironment{qsolbox}
{
  \begin{center}
  \begin{minipage}{0.92\linewidth}
  \hrule\vspace{0.75em}
}
{
  \vspace{0.75em}\hrule
  \end{minipage}
  \end{center}
}

%------------------------------------------------------------------------------
% End of Macros
%==============================================================================


